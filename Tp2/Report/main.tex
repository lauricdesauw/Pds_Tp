\documentclass{article}
\usepackage[utf8]{inputenc}
\usepackage{graphicx}
\usepackage{fullpage}



\title{A front compiler for VSL+}
\author{Yan Garito, Lauric Desauw}
\date{December 2019}

\begin{document}

\maketitle

\section{Introduction}
\section{General structure}
\section{Expression}

\section{Bloc}

\section{Control structure}
\subsection{If Then Else}
\subsection{While}

\section{Tabular}

\section{Functions}
Speak about the ret void 
\section{The PRINT and READ}

\begin{itemize}
\item[\underline{Print}:] We have defined an instruction \emph{PrintInstruction}
  wich take an list of \emph{printable} which is a wrapper for string and
  expression.
  We generate the string we want to print by replacing all \emph{int} by the
  string \emph{``\%d''} and we save all those variables and their order for the next step.
  This string is put in the header in order as a global variable. 
  Then we make a call to the function \emph{@printf} with the previous string and the variables we wants to print. 
\item[\underline{Read}:] The method is the same as previously but instead of
  \emph{printable} the instruction take \emph{varaibles} and in the llvm we call \emph{@scanf}. 
\end{itemize}

\section{Pointer and Structure}
As an extension we choosed to implemente pointer and structure in our language. First the pointer are usefull
when we want a function to return more than one argument, we can put all of them in the pointer and return it.
Then the structure have similar use but you do not need to know the place of the parameter you want, only it's name, which is
often easier to memorize.

In this section we are explaining how we implemented those two objects, the limitations of our implementation and
some way to improve it.

\subsection{Implementation of Pointer}
\subsection{Implementation of Structure}




\end{document}